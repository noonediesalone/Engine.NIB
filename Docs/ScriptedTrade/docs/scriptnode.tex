A script is described by

\begin{itemize}
\item the script code (in the \verb+Code+ node)
\item the name of the variable used to populate the instrument's NPV result field (in the \verb+NPV+ node)
\item an optional list of variables used to populate the additional result map in an instrument
\item an optional specification of calibration strikes, see \ref{calibration} for details
\item an optional ScheduleCoarsening node specifying which schedules can be coarsened, see \ref{grid_coarsening} for details
  on this
\item an optional NewSchedules node specifying new schedules should be created before the script execution, this node
  contains
  \begin{itemize}
  \item Name: a name for the new schedule to be created
  \item Operation: an operation to perform, only Join is supported currently
  \item Schedules: a list of source schedules
  \end{itemize}
\item an optional StickyCloseOutStates node specifying variables that should be held constant during an AMC exposure run
  with sticky close-out mpor mode, usually a list of exercise / barrier hit indicators, see \ref{scripting_amc} for more
  details on this
\item an optional ConditionalExpectation node specifying a filter on model states to be used in NPV() and NPVMEM()
  functions. The filter is specified as a subnode ModelStates with one or several ModelState subnodes ``Asset'' (use EQ,
  FX, COMM components), ``IR'' (use interest rate states), ``INF'' (use inflation states). Applies to GaussianCam model
  only. If left empty, the full model state is used for conditional npv calculation.
\end{itemize}

Several script nodes can be used in parallel and are distinguished by an optional \verb+purpose+ attribute then. There
must always be a script with empty purpose. Special purposes are

\begin{itemize}
  \item \verb+FD+: If a script with purpose FD is available this is preferred over the default script when an FD engine
    is built (as opposed to an MC engine)
  \item \verb+AMC+: If a script with purpose AMC is available this is preferred over the default script when an AMC
    engine is built, i.e. an engine used within the AMC analytics type. See \ref{scripting_amc} for more details.
\end{itemize}

The keys in the instrument's additional results map are by default identical to the variable names used to populate
them. They can be given a different name using the optional rename attribute. The variables can be scalars or arrays of
any type which will be translated to QuantLib instrument result types \verb+double+ (for NUMBER), \verb+QuantLib::Date+
(for EVENT), \verb+string+ (for INDEX, CURRENCY, DAYCOUNTER) for scalars or vectors thereof for arrays.

The following additional results have a special meaning.

\begin{itemize}
\item \verb+currentNotional+ a number representing the current notional of a trade, if given this is displayed in the
  NPV report and - importantly - required as an input for trades which fall under the IM Schedule approach
\item \verb+notionalCurrency+ the currency in which the currentNotional is given
\end{itemize}

\begin{minted}[fontsize=\footnotesize]{xml}
    <Script purpose="">
      <Code><![CDATA[
             NUMBER Payoff, ExerciseProbability;
             Payoff = PutCall * (Underlying(Expiry) - Strike);
             Option = LongShort * Quantity * PAY( max( Payoff, 0 ), Expiry, Settlement, PayCcy);
             IF Payoff > 0.0 THEN
                 ExerciseProbability = 1;
             END;
        ]]></Code>
      <NPV>Option</NPV>
      <Results>
        <Result>ExerciseProbability</Result>
        <Result>currentNotional</Result>
        <Result rename="notionalCurrency">PayCcy</Result>
      </Results>
      <CalibrationSpec>
        <Calibration>
          <Index>Underlying</Index>
          <Strikes>
            <Strike>Strike</Strike>
          </Strikes>
        </Calibration>
      </CalibrationSpec>
      <ScheduleCoarsening/>
      <NewSchedules>
        <NewSchedule>
          <Name>ExerciseAndSimDates</Name>
          <Operation>Join</Operation>
          <Schedules>
            <Schedule>_AMC_SimDates</Schedule>
            <Schedule>ExerciseDates</Schedule>
          </Schedules>
        </NewSchedule>
      </Newschedules>
      <StickyCloseOutStates>
        <StickyCLoseOutState>ExerciseIndicator</StickyCloseOutState>
      </StickyCloseOutstates>
      <ConditionalExpectation>
        <ModelStates>
          <ModelState>Asset</ModelState>
        </ModelStates>
      </ConditionalExpectation>
    </Script>
\end{minted}
