

% \ifdefined\UserGuidePlus\newcommand{\smsection}{\section}\fi
% \ifdefined\UserGuidePlus\newcommand{\smsubsection}{\subsection}\fi
% \ifdefined\UserGuide\newcommand{\smsection}{\section}\fi
% \ifdefined\UserGuide\newcommand{\smsubsection}{\subsection}\fi
% \ifdefined\RiskCatalogue\newcommand{\smsection}{\section}\fi
% \ifdefined\RiskCatalogue\newcommand{\smsubsection}{\subsection}\fi
% \ifdefined\ProductCatalogueBuild\newcommand{\smsection}{\section}\fi
% \ifdefined\ProductCatalogueBuild\newcommand{\smsubsection}{\subsection}\fi
%========================================================
\section{Structured Messages}\label{sec:structured_messages}
%========================================================

% Intro to the concept of structured messages, and that there are structured 'warnings' and structured 'errors'

When any part of an anlytic or set of calculations fails, a corresponding error or warning message is logged, which we call
a \lstinline!StructuredMessage!. Where possible, ORE will attempt to complete the
rest of the calculation, e.g.\ where the error pertains to a specific netting set or trade, then the output reports might still be
produced with the correct values for all other netting sets or trades, and a \lstinline!StructuredMessage! will provide the user with
information to help with debugging and fixing the issue.

A structured message has the following form:

\begin{minted}{json}
  {
    "category": "Error",
    "group": "Trade",
    "message": "...",
    "subFields": [
      {
        "fieldName": "...",
        "fieldValue": "..."
      },
      {
        "fieldName": "...",
        "fieldValue": "..."
      }
    ]
  }
\end{minted}

% Structured Message Categories

\subsection{Categories}

There are two main types/categories of structured messages.
\begin{enumerate}
  \item \textbf{Warnings} - Structured warning messages are returned when there is an apparent inconsistency in input values, or when
  amounts are unexpectedly large (or small). If the user intended this, then there may be nothing wrong with the calculation and the
  output can be assumed to be correct.
  \item \textbf{Errors} - Structured error messages are returned when part of a calculation has failed, or when something has gone
  wrong to the extent that continuing the calculation would yield nonsensical results, and hence the calculation is either skipped
  over or failed entirely. ORE will always attempt to complete the calculation so that only results from the failing components are
  missing.
\end{enumerate}
Where the distinction between the two can sometimes be unclear, a structured warning will only ever be raised for apparent
inconsistencies or unusual values, while a structured error would indicate that a component has failed. Depending on the circumstances,
these can either lead to a larger failure later on in the calculations, or sometimes have no impact on the output reports.

% Structured Message Groups

\subsection{Groups}

Within each category (i.e.\ warning or error), structured messages are further subdivided into groups, which give an indication of what
part of ORE is failing. A structured error (warning) has a corresponding \lstinline!message! and usually an \lstinline!exceptionType!
(\lstinline!warningType!) subfield. Additionally, there may be other subfields based on the structured message group, e.g.\ for
structured trade error messages, the trade ID and trade type:

\begin{itemize}
  \item Analytics: \lstinline!analyticType!
  \item Configuration: \lstinline!configurationType!, \lstinline!configurationId!
  \item Model
  \item Curve: \lstinline!curveId!
  \item Trade: \lstinline!tradeId!, \lstinline!tradeType!
\end{itemize}


% Specific Instances of Warnings/Examples
