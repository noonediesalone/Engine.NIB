\subsubsection{Flexi Swap}

A Flexi Swap is an amortizing swap in which one party has the option to further 
reduce the notional in each period to each value between the current notional and 
a specified lower bound for that period.

Flexi Swaps are priced in ORE following the Replication Approach described in (F. Jamshidian, 2005). The replicating basket of
Bermudan Swaptions is priced using the method as described under Product Type ``Bermudan Swaption'', but using one
global calibration for all swaptions derived from the flexi swap structure.
Prepayment option types ``ReductionByAbsoluteAmount'' and ``ReductionUpToAbsoluteAmount'' are both treated approximately
by generating a schedule of deterministic, lower notional bounds from assuming that all earlier prepayment options
were executed.

The \lstinline!FlexiSwapData! node is the trade data container for trade type Flexi Swap. A Flexi Swap is a two-legged
swap with optional and customisable pre-payments. Flexi Swaps are typically used for representing swaps linked to Asset
Backed Securities with flexible amortisation. A Flexi Swap must have two legs, one fixed and one floating. The floating
leg must have a pay frequency that is a multiple of the fixed leg frequency and corresponding floating and fixed leg
periods must have the same notional. The legs typically have an amortising notional and are represented by
\lstinline!LegData! trade component sub-nodes, described in section \ref{ss:leg_data}. The \lstinline!FlexiSwapData!
node also contains a \lstinline!OptionLongShort! node indicating the holder of the prepayment option and a node
describing the optional prepayments, see below.

An example structure of a \lstinline!FlexiSwapData! node is shown in Listing \ref{lst:flexiswap_data}. In this case the
optional pre-payments are given by a subnode \lstinline!LowerNotionalBounds!  meaning that the notional of the swap can
be reduced to any value between the given lower bound and the original notional in each fixed leg period.

\begin{listing}[H]
%\hrule\medskip
\begin{minted}[fontsize=\footnotesize]{xml}
<FlexiSwapData>
  <LowerNotionalBounds>
        <Notional>451389557.145667</Notional>
        <Notional>427876791.621303</Notional>
        <Notional>404435982.369285</Notional>
        <Notional>379353200.32956</Notional>
        ...
  </LowerNotionalBounds>
  <OptionLongShort>Short</OptionLongShort>
  <LegData>
    <LegType>Fixed</LegType>
    ...
  </LegData>
  <LegData>
    <LegType>Floating</LegType>
    ...
  </LegData>
</FlexiSwapData>
\end{minted}
\caption{Flexi Swap data}
\label{lst:flexiswap_data}
\end{listing}

Alternatively the optional pre-payments can be described by a subnode \lstinline!NotionalDecreases! which is more
general than the description via LowerNotionalBounds (using the reduction type RedutionToLowerBound, see below for more
details on this), see Listing \ref{lst:flexiswap_data2} for an example.

\begin{listing}[H]
%\hrule\medskip
\begin{minted}[fontsize=\footnotesize]{xml}
<FlexiSwapData>
  <Prepayment>
    <NoticePeriod>5D</NoticePeriod>
    <NoticeCalendar>TARGET</NoticeCalendar>
    <NoticeConvention>F</NoticePeriod>
    <PrepaymentOptions>
      <PrepaymentOption>
        <ExerciseDate>2015-02-01</ExerciseDate>
        <Type>ReductionUpToLowerBound</Type>
        <Value>404435982.369285</Value>
      </PrepaymentOption>
      <PrepaymentOption>
        <ExerciseDate>2016-02-01</ExerciseDate>
        <Type>ReductionByAbsoluteAmount</Type>
        <Value>100000.0</Value>
      </PrepaymentOption>
      <PrepaymentOption>
        <ExerciseDate>2017-02-01</ExerciseDate>
        <Type>ReductionUpToAbsoluteAmount</Type>
        <Value>50000.0</Value>
      </PrepaymentOption>
    <PrepaymentOptions>
  </Prepayment>
  <OptionLongShort>Short</OptionLongShort>
  <LegData>
    <LegType>Fixed</LegType>
    ...
  </LegData>
  <LegData>
    <LegType>Floating</LegType>
    ...
  </LegData>
</FlexiSwapData>
\end{minted}
\caption{Flexi Swap data}
\label{lst:flexiswap_data2}
\end{listing}

The meanings and allowable values of the elements in the \lstinline!FlexiSwapData!  node follow below.

\begin{itemize}

\item OptionLongShort: Specifies which party has the right to pre-pay the notional down to the lower notional
  bound. \emph{Short} means that for pricing purposes pre-payments are assumed to be done in such a way to maximise the
  value of the Flexi Swap for the ``other'' counterparty, \emph{Long} means that the Flexi Swap value is maximised from
  ``our'' point of view.

Allowable values: \emph{Long} or \emph{Short}

\item LegData: This is a trade component sub-node outlined in section \ref{ss:leg_data}. A Flexi Swap must have two
  \lstinline!LegData! nodes and the LegType element must be set to \emph{Floating} on one leg and \emph{Fixed} on the
  other. The two legs must have the same \lstinline!Currency!. The float leg pay frequency must be a multiple of the
  fixed leg frequency.

\end{itemize}

The optional prepayments are described by either a \verb+LowerNotionalBounds+ node or a \verb+Prepyment+ node.

In case the optional prepayments are described by a \verb+LowerNotionalBounds+ node, the minimum level to which the
notional can be amortised down to must be given as a notional schedule. The schedule can be specified as described in
\ref{ss:leg_data}, i.e. using a sequence of \verb+Notional+ subnodes or using the \verb+startDate+ attribute to specify
notional changes. The given schedule must be given for the fixed leg periods since the notional can be decreased for
each whole fixed leg period and the corresponding floating leg periods (remember that the floating leg frequency must be
a multiple of the fixed leg frequency). Each lower notional bound child element can take a positive real number that
cannot exceed the notional amount of the corresponding coupon period on either leg and (from the second fixed coupon
period on) the lower notional bound of the previous coupon period.

In case the optional prepayments are described by a \verb+Prepayment+ node, the the single exercise opportunities are
described by a \verb+PrepaymentOptions+ subnode that contains one or several \verb+PerpaymentOption+ subnodes, each of
which comprises the following elements:

\begin{itemize}
\item ExerciseDate: The date on which the notional can be decreased.
\item Type: The type of the allowed notional reduction. The allowable types are
  \begin{itemize}
    \item ReductionUpTpLowerBound: The notional can be reduced to any value between the current notional and the
      lower bound given in the Value node.
    \item ReductionByAbsoluteAmount: The notional can be reduced by an absolute amount given in the Value node. If this
      value is greater than the current notional, the reduction amount is equal to the current notional.
    \item ReductionUpToAbsoluteAmount: The notional can be reduced by any value between zero and a
      given absolute amount (given in the Value node).
  \end{itemize}
\item Value: The value that together with the type describes the amount by which the notional can be decreased.
\end{itemize}

In addition the \verb+Prepayment+ node contains the following optional subnodes describing the conventions for deriving
the option notice date from the exercise date:

\begin{itemize}
\item NoticePeriod [Optional]: The notice period defining the date (relative to the exercise date) on which the exercise
  decision has to be taken. If not given the notice period defaults to 0D, i.e. the notice date is identical to the
  exercise date.
\item NoticeCalendar [Optional]: The calendar used to compute the notice date from the exercise date. If not given
  defaults to the null calendar (no holidays, weekends are no holidays either).
\item NoticeConvention [Optional]: The convention used to compute the notice date from the exercise date. Defaults to
  Unadjusted if not given.
\end{itemize}
