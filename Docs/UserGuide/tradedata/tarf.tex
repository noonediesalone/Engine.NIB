\subsubsection{Target Redemption Forward (TaRF)}
\label{sec:tarf}

\ifdefined\IncludePayoff{{\bf Payoff}

A TaRF is a sequence of (FX or Equity or Commodity) forward contracts, specified by some strike(s) $K_k$ and leverage
factors $L_k$ applicable to ranges $[A_k, B_k]$, $k=1,\ldots,K$, for the underlying price $X_i$ at a sequence of expiry
dates $T_i$. If a $T_i$ lies in the future, the associated fixing $X_i$ is not known at the valuation date.

The trade accumulates positive profits

$$
P_i = P_{i-1} + \sum_{k=1}^K N\, L_k\, \max(X_i - K_k, 0)\, 1_{A_k < X_i \leq B_k}, \qquad i \geq 1, \qquad P_0 = 0
$$

and terminates (knocks out) under some conditions on $P_i$:

There are two types of knock out
\begin{itemize}
\item A Continuous TaRF terminates when the profit reaches or exceeds a
pre-determined target.
\item A Digital TaRF terminates when the number of fixing days where the
buyer makes a profit reaches a pre-determined target.
\end{itemize}

The payment after the knock out event can be of different types
\begin{itemize}
\item exact: the payment is adjusted to match the pre-determined target
\item full: the unadjusted payment is made
\item truncated: no payment is made
\end{itemize}

Additional features include
\begin{itemize}
\item European Knock Ins (EKI), where no sells or buys happen if a fixing is between the strike
and the EKI level
\item Pivots, where different strikes apply for different ranges of the fixing
\end{itemize}

{\bf Input}}\fi

The \verb+FxTaRFData+, \verb+EquityTaRFData+, \verb+CommodityTaRFData+ is the trade data container for the FxTaRF,
EquityTaRF, CommodityTaRF trade type, listings \ref{lst:fxtarf_data} and \ref{lst:eqtarf_data} show the structure of
example trades for an FX and Equity underlying.

\begin{listing}[H]
\begin{minted}[fontsize=\footnotesize]{xml}
<Trade id="FX_TARF">
  <TradeType>FxTaRF</TradeType>
  <Envelope/>
  <FxTaRFData>
    <Currency>USD</Currency>
    <FixingAmount>1000000</FixingAmount>
    <TargetAmount>100000</TargetAmount>
    <Strike>1.1</Strike>
    <Underlying>
      <Type>FX</Type>
      <Name>ECB-EUR-USD</Name>
    </Underlying>
    <ScheduleData>
      <Dates>
        <Dates>
          <Date>2017-03-01</Date>
          <Date>2020-03-01</Date>
          <Date>2025-03-01</Date>
          <Date>2029-03-01</Date>
        </Dates>
      </Dates>
    </ScheduleData>
    <OptionData>
      <LongShort>Long</LongShort>
      <PayoffType>TargetExact</PayoffType>
    </OptionData>
    <RangeBounds>
      <RangeBound>
        <RangeTo>1.05</RangeTo>
        <Leverage>2</Leverage>
      </RangeBound>
      <RangeBound>
        <RangeFrom>1.1</RangeFrom>
        <Leverage>1</Leverage>
      </RangeBound>
    </RangeBounds>
    <Barriers>
      <BarrierData>
        <Type>CumulatedProfitCap</Type>
        <Levels>
          <Level>100000</Level>
        </Levels>
      </BarrierData>
    </Barriers>
  </FxTaRFData>
</Trade>
\end{minted}
\caption{FxTaRF data}
\label{lst:fxtarf_data}
\end{listing}

\begin{listing}[H]
\begin{minted}[fontsize=\footnotesize]{xml}
<Trade id="EQ_TARF">
  <TradeType>EquityTaRF</TradeType>
  <Envelope/>
  <EquityTaRFData>
    <Currency>USD</Currency>
    <FixingAmount>775</FixingAmount>
    <Strike>2900</Strike>
    <Underlying>
      <Type>Equity</Type>
      <Name>RIC:.SPX</Name>
    </Underlying>
    <ScheduleData>
      <Dates>
        <Dates>
          <Date>2017-03-01</Date>
          <Date>2020-03-01</Date>
          <Date>2025-03-01</Date>
          <Date>2026-03-01</Date> 
          <Date>2027-03-01</Date>                   
          <Date>2028-03-01</Date>
        </Dates>
      </Dates>
    </ScheduleData>
    <OptionData>
      <LongShort>Long</LongShort>
      <PayoffType>TargetFull</PayoffType>
    </OptionData>
    <RangeBounds>
        <RangeBound>
            <RangeTo>3300</RangeTo>
            <Leverage>1</Leverage>
        </RangeBound>
        <RangeBound>
            <RangeFrom>3300</RangeFrom>
            <Leverage>2</Leverage>
            <StrikeAdjustment>100</StrikeAdjustment>
        </RangeBound>
    </RangeBounds>
    <Barriers>
      <BarrierData>
        <Type>FixingCap</Type>
        <Levels>
          <Level>3</Level>
        </Levels>
      </BarrierData>
    </Barriers>
  </EquityTaRFData>
</Trade>
\end{minted}
\caption{EquityTaRF data}
\label{lst:eqtarf_data}
\end{listing}

The payout formula for each fixing date of a TaRF - for a Long position - is:
$
Payout = RangeBound(Leverage) \cdot FixingAmount \cdot (fix - Strike(global/local))
$


The meanings and allowable values of the elements in the data  node follow below.

\begin{itemize}
    \item Currency: The payout currency of the TaRF. The TargetAmount and the result of the payout formula above are
expressed in this currency.  For FX, this should be \lstinline!CCY2! as defined in the Name field in the Underlying node.
Note that for (non-quanto) FxTaRFs this should be the domestic (\lstinline!CCY2!) currency, as defined in the Name field in
the Underlying node. For non-quanto Equity- and CommodityTaRFs this should be the currency the equity or commodity is quoted in.
Notice section ``Payment Currency'' in ore/Docs/ScriptedTrade. \\
      Allowable values: See Table \ref{tab:currency} \lstinline!Currency!.

    \item FixingAmount: The unlevered FixingAmount amount to be used at each fixing date, as per the payout formula above.  
    
    For FxTaRFs: The FixingAmount is expressed in the foreign currency (\lstinline!CCY1!).
    
   For EquityTaRFs: The FixingAmount is expressed as number of shares/units of the underlying equity or equity index.
    
   For CommodityTaRFs: The FixingAmount is expressed as number of units of the underlying commodity.
    
      Allowable values: Any positive real number. Note that the FixingAmount must
be positive both for Long and Short positions.

    \item TargetAmount[Optional]: If PayoffType = \emph{TargetExact}, after a knock-out event the last payout is
      adjusted so that the cumulated profit since the start of the TaRF matches the target amount.  TargetAmount is
      optional, only required for PayoffType = \emph{TargetExact}. The TargetAmount is a currency amount, for FxTaRFs it
      is in (\lstinline!CCY2!).

      Notice that if the barrier type is \emph{CumulatedProfitCapPoints} the last payout is adjusted in a way such that
      the cumulated profit in relative terms (as defined for this barrier type below) matches the target amount divided
      by the fixing amount (i.e. the target amount in points). \\

      Allowable values: Any positive real number.

    \item TargetPoints[Optional]: For PayoffType = \emph{TargetExact}, this provides an alternative way of expressing
      the TargetAmount as a percentage of FixingAmount in decimal form. For example a FixingAmount = $1,000,000$ and
      TargetPoints = $0.0700$ translates to a TargetAmount = $70,000$. \\

      Allowable values: Any positive real number.

    \item Strikes [Optional]: Global strikes associated to the ranges. Either this node or the Strike node must be given
      to specify (a) global strike(s) valid across all ranges. If no such global strikes are given, the strikes have to
      be specified as part of the range bound definition.  The Strikes node represents time-varying strikes while the
      Strike node represents a strike that is constant over time. If the Strikes node is given, it must contain a number
      of Strike subnodes. Either
     \begin{itemize}
       \item the Strike subnodes except the first one must contain an attribute startDate which indicates from which
         fixing date on the strike is valid. The first strike is valid until the minimum startDate given in the other
         nodes, or
       \item the values in the Strike subnodes are associated to the given fixing dates in their order, the number of
         values must be equal to the number of fixing dates or less, in the latter case the last given value is repeated
         to match the number of fixing dates.
     \end{itemize}
     See the description under Strike for further information on the single strike values and the relation to strike
     information given in range bounds.

    \item Strike [Optional]: Global strike associated to the ranges. Is overwritten by local strikes defined in
      \lstinline!RangeBounds! or modified by range strike adjustments. Note that each RangeBound needs a strike, either
      the global strike defined here, or a local one in the \lstinline!RangeBounds! nodes which then overwrites the
      global one. For FX, the strike is expressed as amount in \lstinline!CCY2! per one unit of \lstinline!CCY1! as
      defined in the Name field in the Underlying node.
    
The logic for global and local strikes is as follows:

- if a local Strike for a range is given, this local Strike will be used for the range (a global Strike will be ignored if given)\\
- otherwise if a global Strike and a local StrikeAdjustment are given, the strike used for the range will be global Strike + local StrikeAdjustment\\
- otherwise if a global Strike, but no local StrikeAdjustment is given, the strike used for the range will be the global Strike\\
- if no global strike and no local strike is given for an range, an error is thrown, since the strike information is not sufficient
    
      Allowable values: Any positive real number. 
    \item Underlying: A node with the underlying of the TaRF instrument.

For FxTaRFs: \lstinline!Type! is set to \emph{FX} and \lstinline!Name! is a string on the form \emph{SOURCE-CCY1-CCY2} where \lstinline!CCY1! is the foreign currency, \lstinline!CCY2! is the domestic currency, and \emph{SOURCE} is the fixing source, see Table \ref{tab:fxindex_data} and  \ref{ss:underlying}.

For EquityTaRFs: \lstinline!Type! is set to \emph{Equity} and \lstinline!Name! and other fields are as outlined in \ref{ss:underlying}

For CommodityTaRFs: \lstinline!Type! is set to \emph{Commodity} and \lstinline!Name! is an identifier of the commodity as outlined in \ref{ss:underlying} and in Table \ref{tab:commodity_data}.

      Allowable values: Any FX, Equity or Commodity underlying as specified in \ref{ss:underlying}
    \item ScheduleData: The fixing dates schedule of the TaRF.\\
      Allowable values: See \ref{ss:schedule_data}
    \item SettlementLag [Optional]: The settlement delay. \\
      Allowable values: Any period definition (e.g.\ \emph{2D}, \emph{1W}, \emph{1M}, \emph{1Y}). If omitted or left blank, defaults to \emph{0D}.
    \item SettlementCalendar [Optional]: The calendar used to compute the settlement date from the corresponding fixing date. \\
      Allowable values: Any valid calendar, see Table \ref{tab:calendar} Calendar. If omitted or left blank, defaults to the \emph{NullCalendar} (no holidays).
    \item SettlementConvention [Optional]: The convention used to compute the settlement date from the corresponding fixing date. \\
      Allowable values: Any valid roll convention (\emph{U,F,P,MF,MP}), see Table \ref{tab:convention} Roll Convention. If omitted or left blank, defaults to \emph{F} (following).
    \item OptionData: This is a trade component sub-node outlined in section \ref{ss:option_data}. 
The relevant fields in the \lstinline!OptionData! node for an TaRF are:

\begin{itemize}
\item \lstinline!LongShort! The allowable values are \emph{Long} or \emph{Short}.
\item \lstinline!PayoffType! can be \emph{TargetFull}, \emph{TargetExact}, or \emph{TargetTruncated}, impacting the payment after a Knock-Out event as per:
      
     \emph{TargetExact}: the payment is set to the pre-determined \lstinline!TargetAmount!\\
     \emph{TargetFull}: the unadjusted payment is made\\
     \emph{TargetTruncated}: no payment is made
\end{itemize}

    \item RangeBoundSet: Time-varying range-bound information. Either this node or the RangeBounds node (for time-independent range bounds) must be given. If the RangeBoundSet node is given, it must contain a number of RangeBounds nodes. Either
     \begin{itemize}
       \item the RangeBounds subnodes except the first one must contain an attribute startDate which indicates from
         which fixing date on the values are valid. The first value is valid until the minimum startDate given in the
         other nodes, or
       \item the values in the RangeBounds subnodes are associated to the given fixing dates in their order, the number
         of RangeBounds subnodes must be equal to the number of fixing dates or less, in the latter case the last given
         RangeBounds subnode is repeated to match the number of fixing dates.
     \end{itemize}
     See the description under RangeBounds for further information on the single RangeBounds nodes.

    \item RangeBounds: Nodes for the specification of ranges (intervals) and associated \lstinline!Leverage!, local
      \lstinline!Strike! and local \lstinline!StrikeAdjustment!. A range/interval can be specified to apply for all
      values the underlying can take, or for specific ranges using \lstinline!RangeFrom! and
      \lstinline!RangeTo!. Multiple \lstinline!RangeBound! sub-nodes can be included within the \lstinline!RangeBounds!
      node. If the value for \lstinline!RangeFrom! is omitted it is interpreted as $-\infty$ (unbounded towards minus
      infinity). If \lstinline!RangeTo! is omitted it is interpreted as $\infty$ (unbounded towards plus infinity).
      Note that if the underlying takes a value not covered by any range/interval, no payout will occur.
    
    For FxTaRFs: The the  \lstinline!Strike!, \lstinline!StrikeAdjustment!, \lstinline!RangeFrom! and \lstinline!RangeTo! Fx rates are all defined as amount in domestic
currency (\lstinline!CCY2!) for one unit of foreign currency (\lstinline!CCY1!).
    
For Equity- and CommodityTaRFs: The \lstinline!Strike!, \lstinline!StrikeAdjustment!, \lstinline!RangeFrom! and \lstinline!RangeTo!  are all defined as the value of one
unit/share/contract of the underlying equity or commodity, expressed in the currency the equity or commodity is quoted in. \\
    
Allowable values: For each range, see \ref{ss:rangebound}. Note that an interval/range can't have both a local \lstinline!Strike!  and a local \lstinline!StrikeAdjustment!. If no global \lstinline!Strike! is given, each interval/range must have a local \lstinline!Strike!.

    \item Barriers: Specification of barriers, see \ref{ss:barrier_data}. Multiple \lstinline!BarrierData! sub-nodes can be included within the \lstinline!Barriers! node. Relevant fields for each TaRF \lstinline!BarrierData! sub-node are \lstinline!Type! and \lstinline!Level!.  For a TaRF, \lstinline!Type! can be set to \emph{CumulatedProfitCap}, \emph{CumulatedProfitCapPoints} or \emph{FixingCap}. Notice that \emph{CumulatedProfitCapPoints} can not be combined with the other barrier types.
    
    \begin{itemize}
    \item \emph{CumulatedProfitCap}: The TaRF terminates once the generated profit from the long side's view reaches the
      CumulatedProfitCap. (Continuous TaRF) For an FxTaRF the CumulatedProfitCap is quoted in
      \lstinline!CCY2!. \\ Allowable values: The \lstinline!Level! for a \emph{CumulatedProfitCap} can be set to any
      non-negative real number.

    \item \emph{CumulatedProfitCapPoints}: The TaRF terminates once the generated profit from the long side's view
      reaches the CumulatedProfitCapPoints. (Continuous TaRF) Here, the generated profit is measured as the absolute
      profit amount (as used in CumulatedProfitCap) divided by the FixingAmount and divided by the absolute value of the
      leverage. \\ Allowable values: The \lstinline!Level! for a \emph{CumulatedProfitCapPoints} can be set to any
      non-negative real number.

    \item \emph{FixingCap}: The TaRF terminates once number of fixings where a profit is generated - from the long side's view -reaches the FixingCap. (Digital TaRF)\\
     Allowable values: The \lstinline!Level! for a \emph{FixingCap} can be set to a non-negative integer.
    \end{itemize}

\end{itemize}

% scripted trade representation

Target Redemption Forwards can alternatively represented as {\em scripted trades},
refer to the scripted trade documentation in ore/Docs/ScriptedTrade
for an introduction. This representation does not allow for time varying strikes or range
bounds and also not for a target amount / cumulated profit cap specification in points.

\begin{minted}[fontsize=\footnotesize]{xml}
<Trade id="TaRF#1">
  <TradeType>ScriptedTrade</TradeType>
  <Envelope/>
  <TaRFData>
    <FixingAmount type="number">1000000</FixingAmount>
    <LongShort type="longShort">Long</LongShort>
    <Underlying type="index">FX-ECB-EUR-USD</Underlying>
    <PayCcy type="currency">USD</PayCcy>
    <FixingDates type="event">
      <ScheduleData>
        <Dates>
          <Dates>
              <Date>2017-03-01</Date>
              <Date>2020-03-01</Date>
              <Date>2025-03-01</Date>
              <Date>2029-03-01</Date>
          </Dates>
        </Dates>
      </ScheduleData>
    </FixingDates>
    <SettlementDates type="event">
      <DerivedSchedule>
        <BaseSchedule>FixingDates</BaseSchedule>
        <Shift>2D</Shift>
        <Calendar>TARGET</Calendar>
        <Convention>F</Convention>
      </DerivedSchedule>
    </SettlementDates>
    <RangeUpperBounds type="number">
        <Value>1.05</Value>
        <Value>10000</Value>
    </RangeUpperBounds>
    <RangeLowerBounds type="number">
        <Value>0</Value>
        <Value>1.1</Value>
    </RangeLowerBounds>
    <RangeLeverages type="number">
        <Value>2</Value>
        <Value>1</Value>
    </RangeLeverages>
    <RangeStrikes type="number">
      <Value>1.1</Value>
      <Value>1.1</Value>
    </RangeStrikes>
    <KnockOutProfitAmount type="number">100000</KnockOutProfitAmount>
    <KnockOutProfitEvents type="number">0</KnockOutProfitEvents>
    <TargetAmount type="number">100000</TargetAmount>
    <TargetType type="number">0</TargetType>
  </TaRFData>
</Trade>
\end{minted}

The TaRF script referenced in the trade above is shown in Listing \ref{lst:tarf_script}.

\begin{listing}[hbt]
\begin{minted}[fontsize=\footnotesize]{Basic}
REQUIRE FixingAmount > 0;
REQUIRE LongShort == 1 OR LongShort == -1;
REQUIRE SIZE(RangeUpperBounds) == SIZE(RangeLowerBounds);
REQUIRE SIZE(RangeLowerBounds) == SIZE(RangeLeverages);
REQUIRE SIZE(RangeLowerBounds) == SIZE(RangeStrikes);
REQUIRE TargetType == -1 OR TargetType == 0 OR TargetType == 1;
REQUIRE SIZE(FixingDates) == SIZE(SettlementDates);

NUMBER Payoff, d, r, PnL, wasTriggered, AccProfit, Hits, currentNotional;
NUMBER Fixing[SIZE(FixingDates)], Triggered[SIZE(FixingDates)];

FOR r IN (1, SIZE(RangeUpperBounds), 1) DO
  REQUIRE RangeLowerBounds[r] <= RangeUpperBounds[r];
  REQUIRE RangeStrikes[r] >= 0;
END;

FOR d IN (1, SIZE(FixingDates), 1) DO
  Fixing[d] = Underlying(FixingDates[d]);
  IF wasTriggered != 1 THEN
    PnL = 0;
    FOR r IN (1, SIZE(RangeUpperBounds), 1) DO
      IF Fixing[d] > RangeLowerBounds[r] AND Fixing[d] <= RangeUpperBounds[r] THEN
        PnL = PnL + RangeLeverages[r] * FixingAmount * (Fixing[d] - RangeStrikes[r]);
      END;
    END;

    IF PnL >= 0 THEN
      AccProfit = AccProfit + PnL;
      Hits = Hits + 1;
    END;

    IF {KnockOutProfitEvents > 0 AND Hits >= KnockOutProfitEvents} OR
       {KnockOutProfitAmount > 0 AND AccProfit >= KnockOutProfitAmount} THEN
      wasTriggered = 1;
      Triggered[d] = 1;
      IF TargetType == 0 THEN
        Payoff = Payoff + LOGPAY(TargetAmount - (AccProfit - PnL), FixingDates[d], SettlementDates[d], PayCcy, 0, Cashflow);
      END;
      IF TargetType == 1 THEN
        Payoff = Payoff + LOGPAY(PnL, FixingDates[d], SettlementDates[d], PayCcy, 0, Cashflow);
      END;
    ELSE
        Payoff = Payoff + LOGPAY(PnL, FixingDates[d], SettlementDates[d], PayCcy, 0, Cashflow);
    END;
  END;
END;
value = LongShort * Payoff;
currentNotional = FixingAmount * RangeStrikes[1];
\end{minted}
\caption{Payoff script for a TaRF.}
\label{lst:tarf_script}
\end{listing}

The meanings and allowable values of the elements in the \lstinline!TaRFData! node below.

\begin{itemize}
    \item{}[number] \lstinline!FixingAmount!: Amount to buy/sell on each fixing day should be FixingAmount times Leverage of the range that the spot level fixes within. \\
    Allowable values: Any positive real number.
    \item{}[longShort] \lstinline!LongShort!: \emph{Long} if we buy the option. \emph{Short} if we sell the option.
    Allowable values: \emph{Long, Short}
    \item{}[index] \lstinline!Underlying!: Underlying index. \\
    Allowable values: See ore/Docs/ScriptedTrade's Index Section for allowable values.
    \item{}[currency] \lstinline!PayCcy!: The payment currency. For FX, where the underlying is provided
      in the form \lstinline!FX-SOURCE-CCY1-CCY2! (see Table \ref{tab:fxindex_data}) this should
      be \lstinline!CCY2!. If \lstinline!CCY1! or the currency of the underlying (for EQ and
      COMM underlyings), this will result in a quanto payoff. Notice section ``Payment Currency'' in ore/Docs/ScriptedTrade. \\
        Allowable values: See Table \ref{tab:currency} for allowable currency codes.
    \item{}[event] \lstinline!FixingDates!: The fixing dates, given as a ScheduleData node \\
    Allowable values: See section \ref{ss:schedule_data} Schedule Data and Dates, or DerivedSchedule (see the scripted trade documentation in ore/Docs/ScriptedTrade).
    \item{}[event] \lstinline!SettlementDates!: The settlement dates, given as a ScheduleData or DerivedSchedule node.
    Allowable values: See section \ref{ss:schedule_data} Schedule Data and Dates, or DerivedSchedule (see the scripted trade documentation in ore/Docs/ScriptedTrade).
    \item{}[number] \lstinline!RangeUpperBound!: Values of upperbounds for the leverage ranges. If a given range has no upperbound, add 100000. \\
    Allowable values: Any list of positive real numbers. \emph{RangeUpperBound}, \emph{RangeLowerBound} and \emph{RangeLeverages} must have the same size. \emph{RangeUpperBound}[$i$] $\ge$ \emph{RangeLowerBound}[$i$] for all $i$
    \item{}[number] \lstinline!RangeLowerBound!: Values of lowerbounds for the leverage ranges. If a given range has no lowerbound add 0 \\
    Allowable values: Any list of positive real numbers.
    \item{}[number] \lstinline!RangeLeverages!: Values of leverages for the leverage ranges. Positive for Bullish TaRF, Negative for
      Bearish TaRF. \\
    Allowable values: Any list of real numbers.
    \item{}[number] \lstinline!RangeStrikes!: Strikes associated to the ranges. \emph{}. \\
    Allowable values: Any positive real number.
    \item{}[number] \lstinline!KnockOutProfitAmount!: If positive, cap of the profit the buyer can make. \\
    Allowable values: Any positive real number or zero if not applicable.
    \item{}[number] \lstinline!KnockOutProfitEvents!: If positive, cap of the number of fixing days on which buyer makes profit. \\
    Allowable values: Any positive real number or zero if not applicable.
    \item{}[number] \lstinline!TargetAmount!: The target amount to be paid as an accumulated profit for TargetType = exact. \\
    Allowable values: Any positive real number or zero if not applicable.
    \item{}[number] \lstinline!TargetType!: \emph{-1} for truncated, \emph{0} for exact, \emph{1} for full. If DigitalKnockOut is
      true the target type must be set to full. \\
    Allowable values: \emph{-1,0,1}
\end{itemize}
