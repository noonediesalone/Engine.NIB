\subsubsection{Bond}
\label{ss:bond}

A Bond is set up using a {\tt BondData} block, and can be both a stand-alone instrument with trade type \emph{Bond}, or a trade component used by multiple bond derivative instruments.

A Bond can be set up in a short version referencing an underlying bond static, or in a long version where the underlying bond details are specified explicitly, including a full LegData block.
The short version is shown in listing \ref{lst:bonddata_refdata}. The details of the
bond are read from the reference data in this case using the SecurityId as a key. The bond trade is fully specified by

\begin{itemize}
\item SecurityId: The id identifying the bond.

  Allowable Values: A valid bond identifier, typically the ISIN of the reference bond with the ISIN: prefix, e.g.: \verb+ISIN:XXNNNNNNNNNN+
\item BondNotional: The notional of the position in the reference bond, expressed in the currency of the bond.

  Allowable Values: Any non-negative real number
\item CreditRisk [Optional] Boolean flag indicating whether to show Credit Risk on the Bond product. If set to \emph{false}, the product class will  not be set to \emph{Credit}, and there will be no credit sensitivities. However, if the underlying bond reference is set up without a CreditCurveId - typically for some highly rated government bonds -  the CreditRisk flag will have no impact on the product class and no credit sensitivities will be shown even if CreditRisk is set to \emph{true}. 

  Allowable Values: \emph{true} or \emph{false} Defaults to \emph{true} if left blank or omitted.
\end{itemize}

in this case.

\begin{listing}[H]
%\hrule\medskip
\begin{minted}[fontsize=\footnotesize]{xml}
    <BondData>
      <SecurityId>ISIN:XS0982710740</SecurityId>
      <BondNotional>100000000.0</BondNotional>
      <CreditRisk>true</CreditRisk>
    </BondData>
\end{minted}
\caption{Bond Data}
\label{lst:bonddata_refdata}
\end{listing}

For the long version, the bond details are inlined in the trade as shown in listing \ref{lst:bonddata}. The bond specific elements are

\begin{itemize}
\item IssuerId [Optional]: A text description of the issuer of the bond.  This is for informational purposes and not used for pricing.

Allowable values: Any string. If left blank or omitted, the bond will not have any issuer description.

\item CreditCurveId [Optional]: The unique identifier of the bond. This is used for pricing, and is required for bonds for which a credit - related margin component should be generated, and otherwise left blank. If left blank, the bond (and any bond derivatives using the bond as a trade component) will be  plain IR rather than a IR/CR. 

Allowable values: A valid bond identifier, typically the ISIN of the reference bond with the ISIN: prefix, e.g.: \verb+ISIN:XXNNNNNNNNNN+

%Allowable values: 
%See \lstinline!Name! for credit trades in Table \ref{tab:equity_credit_data}. \\
%via the default curves block in {\tt  todaysmarket.xml}
% \item LGD (optional): If given, this LGD is used for pricing, overriding the default LGD of the default curve
\item SecurityId: The unique identifier of the bond.  This defines the security specific spread to be used for pricing.

Allowable values: A valid bond identifier, typically the ISIN of the reference bond with the ISIN: prefix, e.g.: \verb+ISIN:XXNNNNNNNNNN+
  
\item ReferenceCurveId: The benchmark curve to be used for pricing. This is typically the main ibor index for the currency of the bond, and if no ibor index is available for the currency in question, a currency-specific benchmark curve can be used.

Allowable values: 
For currencies with available ibor indices: \\
An alphanumeric string of the form [CCY]-[INDEX]-[TERM]. CCY, INDEX and TERM must be separated by dashes (-). CCY and INDEX must be among the supported currency and index combinations. TERM must be an integer followed by D, W, M or Y. See Table \ref{tab:indices}. 

For currencies without available ibor indices:  \\
An alphanumeric string, matching a benchmark curve set up in the market data configuration in {\tt  todaysmarket.xml} Yield curves section.

Examples: IDRBENCHMARK-IDR-3M, EGPBENCHMARK-EGP-3M, UAHBENCHMARK-UAH-3M, NGNBENCHMARK-NGN-3M
 
\item SettlementDays: The settlement lag in number of business days applicable to the security.

Allowable values: A non-negative integer.

\item Calendar: The calendar associated to the settlement lag.

Allowable values: See Table \ref{tab:calendar} Calendar.

\item IssueDate: The issue date of the security.

See \lstinline!Date! in Table \ref{tab:allow_stand_data}.

\item PriceQuoteMethod [Optional]: The quote method of the bond. Bond price quotes and historical bond prices (stored as
  ``fixings'') follow this method. Also, the initial price for bond total return swaps follows this method. Defaults to
  PerentageOfPar.

  Allowable values: PercentageOfPar or CurrencyPerUnit

\item PriceQuoteBaseValue [Optional]: The base value for quote method = CurrencyPerUnit. Bond price quotes, historical
  bond prices stored as fixings and initial prices in bond total return swaps are divided by this value. Defaults to
  1.0.

  Allowable values: Any real number.

\end{itemize}

A LegData block then defines the cashflow structure of the bond, this can be of type fixed, floating etc. Note that a LegData block should only be included in the long version. 

\begin{listing}[H]
%\hrule\medskip
\begin{minted}[fontsize=\footnotesize]{xml}
    <BondData>
        <IssuerId>Ineos Group Holdings SA</IssuerId>
        <CreditCurveId>ISIN:XS0982710740</CreditCurveId>
        <SecurityId>ISIN:XS0982710740</SecurityId>
        <ReferenceCurveId>EUR-EURIBOR-6M</ReferenceCurveId>
        <SettlementDays>2</SettlementDays>
        <Calendar>TARGET</Calendar>
        <IssueDate>20160203</IssueDate>
        <PriceQuoteMethod>PercentageOfPar</PriceQuoteMethod>
        <PriceQuoteBaseValue>1.0</PriceQuoteBaseValue>
        <LegData>
            <LegType>Fixed</LegType>
            <Payer>false</Payer>
            ...
        </LegData>
    </BondData>
\end{minted}
\caption{Bond Data}
\label{lst:bonddata}
\end{listing}

The bond trade type supports perpetual schedules, i.e. perpetual bonds can be represented by omitting the EndDate in the
leg data schedule definition. Only rule based schedules can be used to indicate perpetual schedules.

%The bond pricing requires a recovery rate that can be specified per SecurityId in the market data configuration.
