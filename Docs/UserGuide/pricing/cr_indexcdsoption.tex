%--------------------------------------------------------------------------------
\subsubsection{Index CDS Option}
\label{pricing:cr_indexcdsoption}

An Index CDS Option is priced using either

\begin{itemize}
\item a Black-Formula approach or
\item a numerical integration approach.
\end{itemize}

The numerical integration approach avoids some approximations that are applied in the Black-Formula and is therefore
considered to be more accurate in general.

\underline{A. Notation of notionals}

In the following we distinguish the following notionals

\begin{itemize}
\item $N_{uf}$ is the {\em unfactored} notional, i.e. the notional excluding all defaults. This is also the notional
  that is used in the trade xml representation.
\item $N_{td}$ is the {\em trade date} notional, i.e. the notional after accounting for defaults between index inception up to
  the trade date
\item $N_{vd}$ is the {\em valuation date} notional, i.e. the notional after accounting for defaults between index
  inception up to the valuation date
\end{itemize}

We start with the description of the calculation of the Front-End Protection Value (FEP). The calculation of FEP and its
components $\text{FEP}_u$ (unrealized part) and $\text{FEP}_r$ (realized part) is identical for both pricing approaches
(Black-Formula, numerical integration).

\underline{B. Discount Curves}

We distinguish two discount curves:

\begin{itemize}
\item The overnight curve associated to the underlying swap currency. We denote discount factors on this curve as
  $P_{OIS}(0,t)$.
\item The overnight curve associated to the collateral currency of the option trade. We denote discount factors on this
  curve as $P_{OIS,t}(0,t)$
\end{itemize}

Furthermore we introduce the notation $P_{*}(0,t)$ as follows:

\begin{itemize}
\item $P_{*}(0,t) := P_{OIS}(0,t)$, i.e. using the swap currency OIS curve, if the option trade is cash settled
\item $P_{*}(0,t) := P_{OIS,t}(0,t)$, i.e. using the trade collateral OIS curve, if the option trade is physically settled
\end{itemize}

\underline{C. Calculation of the Front-End Protection Value FEP}

The total front-end protection value as of the valuation date is given by the sum of the realized and unrealised FEPs,
discounted from the exercise time $t_e$ on the trade collateral curve:

\begin{equation}
\text{FEP} = P_{OIS,t}(0,t_e) ( \text{FEP}_r + \text{FEP}_u )
\end{equation}

The realized $\text{FEP}_r$ is given by the sum over realised losses for underlyings that are defaulted as of the valuation
date and have an auction date later or qeual to the FEP Start date. The realised loss of an underlying is given by its
(pre-default) notional $N_i$ times one minus its (realised) recovery rate $RR_i$, i.e.

\begin{equation}
\text{FEP}_r = \sum_i (1-RR_i) N_i
\end{equation}

where the sum is taken over the defaulted underlyings. The unrealised $\text{FEP}_u$ is given by

\begin{equation}
\text{FEP}_u = \sum_i (1-RR_i) (1 - S_i(t_e)) N_i
\end{equation}

where $S(t_e)$ is the survival probability until the exercise date and the sum is taken over the non-defaulted
underlyings. The survival probability $S_i(t_e)$ and recovery rate $RR_i(t_e)$ is either calculated

\begin{itemize}
\item using the individual name's credit curve and recovery rate (FepCurve = Underlying in the pricing engine config) or
\item using the index credit curve and recovery rate (FepCurve = Index in the pricing engine config)
\end{itemize}

The pricing proceeds in two different ways depending on the type of the input volatility, which can be a
price-volatility or a spread-volatility:

\underline{D. Pricing using a price-volatility input}

If the input volatility is given as a price volatility, no numerical integration is required, instead a simple
application of the Black-Formula can be used in both pricing engines:

If the strike $K$ is given as a price, we adjust for defaults between trade date and valuation date

\begin{equation}
K_P = 1 - \frac{N_{td}}{N_{vd}} \left( 1 - K \right)
\end{equation}

since the contractual strike amount is expressed w.r.t. the trade date notional by market convention, which can be
higher than the notional at valuation date.

If the strike is given as a spread, the first step is to convert the spread strike $K_s$ to a price strike $K_P$ as
follows:

\begin{equation}
K_P = 1 + \frac{N_{td}}{N_{vd}} \cdot RA(K_s) \cdot (c - K_s)
\end{equation}

where

\begin{itemize}
\item $RA(K_s)$ is the risky strike annuity (see below)
\item $c$ is the coupon rate of the CDS
\end{itemize}

Again, the factor $N_{td} / N_{vd}$ again accounts for the fact that the strike amount is calculated using the risky
strike annuity on the notional at trade date. 

The risky strike annuity $RA(K_s)$ is calculated as follows:

\begin{equation}\label{pricing:cr_indexcdsoption_strike_annuity}
RA(K_s) = \frac{\text{NPV}_{K,cpn} + Acc_{cpn}}{N_{vd} \cdot c \cdot S_K(t_e) \cdot P_{OIS}(0,t_e)}
\end{equation}

where

\begin{itemize}
\item $N_{vd}$ is the underlying index CDS notional\footnote{In fact the choice of the notional does not affect
$RA(K_s)$ since it cancels out, so it can be set to an arbitrary value. For numerical reasons it is set to $1/K_s$ in
the actual calculation.}
\item $\text{NPV}_{K,cpn}$ is the premium leg NPV of a {\em strike CDS} in which the coupon rate is set to the strike
  $K_S$. The NPV is calculated on a flat hazard rate curve that prices the strike CDS to zero. We use the swap currency
  OIS curve $P_{OIS}(0,t)$ for discounting.
\item $Acc_{K,cpn}$ is the accrual rebate of the strike CDS
\item $S_K(t_e)$ is the survival probability until option expiry $t_e$ computed on the flat hazard rate curve that prices
  the strike CDS to zero
\end{itemize}

The FEP-adjusted forward price is calculated as

\begin{equation}\label{pricing:cr_indexcdsoption_fep_adjusted_forward}
F_p = \frac{1 - \text{NPV}}{N_{vd} \cdot P_{*}(0,t_e)} - \frac{\text{FEP}}{N_{vd} \cdot P_{OIS,t}(0,t_e)}
\end{equation}

where NPV is the value of the underlying swap. We use

\begin{itemize}
\item constituent curves and recovery rates (Curve = Underlying in the pricing engine config) or
\item index curve and recovery rate (Curve = Index in the pricing engine config)
\end{itemize}

as default curves / recovery rates. The Index CDS option price is now calculated as

\begin{equation}
\text{NPV}_{opt} = N_{vd} \cdot P_{OIS,t}(0, t_e) \cdot \text{Black} \left( K_P, F_p, \sigma\sqrt{t_e}, \omega \right)
\end{equation}

where

\begin{itemize}
\item $\sigma$ is the market volatility for expiry $t_e$ and strike $K_P$
\item $\omega$ is $1$ for a call, $-1$ for a put
\end{itemize}

\underline{E. Pricing using a spread-volatility input}

If the input volatility is a spread-volatility the pricing proceeds depending on the model.

\underline{E.1. Black-Formula}

The risky annuity $RA$ is given by

\begin{equation}\label{pricing:cr_indexcdsoption_risky_annuity}
RA = \frac{\text{NPV}_{cpn} + Acc_{cpn}}{N_{vd} \cdot c \cdot P_{*}(0, t_e) \cdot S(t_e)}
\end{equation}

where

\begin{itemize}
\item $\text{NPV}_{cpn}$ is the premium leg NPV of the CDS. As described above the value is calculated using the
  constituent curves and recovery rates or the index curve and recovery rate depending on the pricing engine
  configuration. The discounting curve is the trade collateral curve for physically settled options resp. the swap
  currency OIS curve for cash settled options.
\item $Acc_{cpn}$ is the accrual rebate of the CDS
\item $S(t_e)$ is the survival probability until option expiry $t_e$
\end{itemize}

The FEP-adjusted forward spread is approximated as a deterministic value

\begin{equation}\label{pricing:cr_indexcdsoption_Sp}
S_p = S + \frac{\text{FEP}}{S(t_e) \cdot RA \cdot N_{vd} \cdot P_{OIS,t}(0,t_e)}
\end{equation}

where $S$ is the fair spread of the underlying index CDS. Note that the numerical integration engine models $S_p$ as a
stochastic quantity on the other hand, see \ref{pricing:cr_indexcdsoption_Sp2}. The adjusted strike spread $K_s'$
is calculated as


\begin{equation}
K_s' = c + \frac{N_{td}}{N_{vd}}\frac{RA(K_s)}{S(t_e) \cdot RA} \left( K_s - c \right)
\end{equation}

where

\begin{itemize}
\item $K_s$ is the spread strike
\item $RA(K_s)$ is the risky strike annuity, see \ref{pricing:cr_indexcdsoption_strike_annuity}
\item $RA$ is the risky annuity, see \ref{pricing:cr_indexcdsoption_risky_annuity}
\end{itemize}

As before, the factor $N_{td} / N_{vd}$ accounts for the fact that the strike adjustment payment upon exercise is
calculated using the risky strike annuity on the notional at trade date rather than the valuation date.

Finally, the option price is calculated using the Black formula

\begin{equation}
  \text{NPV}_{opt} = P_{OIS,t}(0,t_e) \cdot S(t_e) \cdot  RA \cdot N_{vd} \cdot \text{Black} \left( K_s', F_p, \sigma\sqrt{t_e}, \omega \right)
\end{equation}

with a call (protection buyer) / put (protection seller) indicator $\omega = 1, -1$ and a volatility $\sigma$
corresponding to the strike $K$ and expiry $t_e$. The factor $P_{OIS,t}(0,t_e)/P_{*}(0,t_e)$ accounts for possibly
different trade collateral and swap currency OIS curves.

We list two shortcomings of the Black engine compared to the numerical integration engine (see below):

\begin{itemize}
\item the Black engine does not convert a price strike to a spread strike. Instead an error is thrown.
\item the Black engine does not account for defaults between trade and valuation date when reading the pricing
  volatility from the market spread volatility surface
\end{itemize}

\underline{E.2. Numerical Integration}

The first step is to compute the strike adjustment payment $K^*$. In case the trade strike is given as a price strike
$K_P$ this is

\begin{equation}
K^* = \frac{N_{td}}{N_{vd}}(K_P - 1)
\end{equation}

and if the trade strike is given as a spread strike $K_s$ this is

\begin{equation}
K^* = \frac{N_{td}}{N_{vd}} \cdot RA(K_s) \cdot ( c - K_s )
\end{equation}

where $RA(K_s)$ is the risky strike annuity, see \ref{pricing:cr_indexcdsoption_strike_annuity}. As before the factor
$N_{td} / N_{vd}$ accounts for the market convention that the strike adjustment payment should be based on the trade
date notional rather than valuation date notional.

To get the spread strike $K_s$ {\em adjusted for defaults between trade date and valuation date} we use a numerical root
finder to solve

\begin{equation}
RA(K_s) \cdot (c - K_s) = K^*
\end{equation}

for $K_s$. Notice that we can skip the numerical root finding if the trade strike is given as a spread strike {\em and}
$N_{td} = N_{vd}$ (in which case $K_s$ is simply the trade strike).

The FEP-adjusted forward price $F_p$ is calculated as in \ref{pricing:cr_indexcdsoption_fep_adjusted_forward}.  The FEP
adjusted forward spread $S_p$ modeled as a lognormal variable

\begin{equation}\label{pricing:cr_indexcdsoption_Sp2}
S_p(x) = m e^{-\frac{1}{2} \sigma^2 t_e + x \sigma \sqrt{t_e}}
\end{equation}

where $\sigma$ is the pricing volatility for strike $K_s$ and $x \sim N(0,1)$ and $m$ is a free parameter which will be
calibrated in a moment. Compare \ref{pricing:cr_indexcdsoption_Sp2} to the Black pricing
approach \ref{pricing:cr_indexcdsoption_Sp} where we approximated $S_p$ by a deterministic value.

The FEP adjusted index value $V_c(S_p)$ as a function of the FEP-adjusted forward spread $S_p$ is then given as

\begin{equation}
V_c = (S_p - c)\cdot a
\end{equation}

We approximate the true risky annuity by a continuous risky annuity $a$ to speed up calculations. The continuous risky
annuity is defined as

\begin{equation}
a = \int_{t_e}^{t_m} e^{-\int_{t_e}^u (r + \lambda) ds} du
\end{equation}

with interest rate $r$ and instantaneous hazard rate $\lambda$. We set $r$ to the average interest rate over the time
interval $[t_e, t_m]$ from option expiry to underlying maturity and appxoimate $\lambda$ by the usual ``credit
triangle'' $S/(1-R)$ where $R$ is the index recovery rate (used to bootstrap the index CDS curve). This yields

\begin{eqnarray}
r_A &=& \frac{\ln \left( \frac{P_{*}(0,t_m)}{P_{*}(t_e)} \right)}{t_e-t_m} \\
w &=& \left(\frac{S}{1-R} + r \right) \cdot (t_m-t_e) \\
a &=& \left(t_m-t_e\right) \frac{1 - e^{-w}}{w}
\end{eqnarray}

For small $w$, $|w| < 10^{-6}$ we use an expansion to compute $a$ to avoid numerical problems due to $a$ being close to
$0/0$:

\begin{eqnarray}
a = (t_m-t_e) \left(1 - \frac{1}{2}w+\frac{1}{6}w^2-\frac{1}{24}w^3\right)
\end{eqnarray}

The free parameter $m$ from \ref{pricing:cr_indexcdsoption_Sp2} is calibrated to $F_p$
(see \ref{pricing:cr_indexcdsoption_fep_adjusted_forward}) by solving

\begin{equation}
\int_{-\infty}^{\infty} V_c(x,m) \, \phi(x) \, dx = 1 - F_p
\end{equation}

for $m$ numerically. The integration is carried out using the Simpson method with integration bounds $-10 < x < 10$
(i.e. we cover 10 standard deviations of the state variable $x$ in the integration). 

The exercise boundary argument $x^*$ is then computed by numerically solving

\begin{equation}
\left( V_c + K^* + \frac{\text{FEP}_r}{N_{vd}} \right) \phi(x) = 0
\end{equation}

for $x$. Finally the option value is computed as

\begin{equation}
\text{NPV}_{opt} = N_{vd} \cdot P_{OIS,t}(0,t_e) \int_{x_L}^{x_U} \omega \cdot \left(V_c + K^* + \frac{\text{FEP}_r}{N_{vd}}\right) \cdot \phi(x) dx
\end{equation}

with $K_L= x^*, K_U=\infty$ for a call option ($\omega=1$) or $K_L=-\infty, K_U=x^*$ for a put option ($\omega=-1$). As
before, $\infty$ is replaced by $10$ for the actual numerical integration.

\underline{F. Pricing Additional Results}

Table \ref{tab:cr_indexcdsoption_addresults} summarizes the available additional results generated in the various
pricing engines and how they map to the variables used in the previous sections.

\begin{table}[hbt]
  \begin{tabular}{|l|l|}
    \hline
    Variable & Label \\
    \hline
    $\text{FEP}_r$            & realisedFEP \\
    $\text{FEP}_u$            & unrealisedFEP \\
    $\text{FEP}$              & discountedFEP \\
    $N_{td}$                   & tradeDateNotional \\
    $N_{vd}$                   & valuationDateNotional \\
    $K_P$                     & strikePriceDefaultAdjusted \\
    $RA(K_S)/P_{OIS}(0,t_e)$   & forwardRiskyAnnuityStrike \\
    $F_P$                     & fepAdjustedForwardPrice \\
    $F_P$                     & fepAdjustedForwardSpread \\
    $RA$                      & riskyAnnuity \\
    $K_s'$                    & adjustedStrikeSpread \\
    $K_s$                     & strikeSpread \\
    $K^*$                     & strikeAdjustment \\
    $P_{OIS}(0,t_e)$           & discountToExerciseTradeCollateral \\
    $P_{OIS}(0,t_e)$           & discountToExerciseSwapCurrency \\
    $S(x^*)$                  & exerciseBoundary \\
    $\int \ldots dx$          & integrationGrid \\
    $V_c(x)$                  & defaultAdjustedIndexValue \\
    \hline
  \end{tabular}
  \caption{additional results in index cds option pricing engines.}
  \label{tab:cr_indexcdsoption_addresults}
\end{table}

