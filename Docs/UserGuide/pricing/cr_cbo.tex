%--------------------------------------------------------------------------------
\subsubsection{Collateral Bond Obligaton CBO}

We consider an $n$ tranche CBO, or Cashflow CDO. 
The underlying assets consist of a portfolio of corporate bonds or loans with either amortising or bullet structures. 
The portfolio can contain fixed or floating rate obligations. 
Maturities cover a range and do need not coincide. 
We assume that hazard rate and/or security spread data is available and provided externally.

The deal is assumed to be structured as a cashflow securitisation. 
Interest and Notional repayments are directed in an order of priority first to the senior note holder. 
We assume the  available pool for Notional repayments consists of scheduled bond notional repayments and recovery amounts. 
The pool available for interest payments consists of coupons received on the portfolio during the payment period in question.

Class N notes or equity receive the excess pool coupon available after other items in the interest waterfall are discharged.

A typical Interest and Notional waterfall is given in the following.

\bigskip
{\bf Interest Waterfall:}

\begin{enumerate}
\item Senior Fees
\item Interest due on Senior notes
\item Redemption of Senior notes if over-collateralisation or interest coverage tests not met (sufficient to ensure tests are met)
\item Interest due on next most Senior notes
\item Redemption of next most Senior notes if over-collateralisation or interest coverage tests not met (sufficient to ensure tests are met)
\item $\vdots$
\item Subordinated Fees
\item Residual split between management incentive fee and Equity notes
\end{enumerate}

\bigskip

We explicitly write down the pricing formula for each of the CBO notes. First some notation:
\begin{itemize}
\item $t_i$ are the scheduled payment dates of the trade, 
\item $C_{k}(t_i)$  are the coupons paid on the Class $k$ notes on the ith payment date
\item $N_{k}(t_i)$  are the outstanding notionals.
\item $R_k(t_i)$  are the notional redemptions payable on the ith payment date. Note $N_k(t_i)=N_k(t_{i-1})-R_k(t_i)$.
\item $D(t,t_i)$ are the discount factors with respect to the evaluation date $t$.
\item $\Pi_k$ is the present values of the notes which we seek
\end{itemize}

When we take expectations we assume that time to default and interest rates are independent and so all payments are multiplied 
by the Banks risk free discount factor to the payment date and the symbol $\E$ is reserved for expectations under the Credit Measure.

Using these conventions we can write the present values as
\begin{eqnarray}
\Pi_k & = & \sum_{i=1}^n \E(C_k(t_i))D(t,t_i) + \sum_{i=1}^n \E(R_k(t_i))D(t,t_i).
\end{eqnarray}

We will assume in the following that we know the marginal probability of default of each underlying asset 
to any desired date and that we use a standard copula framework (see section \ref{ss:syntheticcdo}) to compute expectations in the presence of joint defaults.

% \subsection*{Implementation}

% The complexity in this product (as opposed to a standard tranched synthetic CDO) comes from the conditional dependence between the interest available for payment and the outstanding notional on which it is to be paid.

% Throughout this section we do not attempt to price the following trade features:
% \begin{itemize}
% \item Event driven termination rights
% \item Purchased or sold termination rights
% \item Management Rights
% \end{itemize}

% We model the trade using a non-trivial modification to  standard Synthetic Unfunded CDO pricing framework as follows:

% \medskip
% First we make the stylised assumptions (for the Loss-Bucketing algorithm)

% \begin{itemize}
% \item The interest waterfall is simplified: All available interest after fixed fees is paid first to the senior note holder and so on down the waterfall with the excess interest paid to the equity note holder.
% \item All principal waterfall is simplified: principal repayments (including recovery amounts) are paid first to the senior note holders and then on down the waterfall to the equity note holders.
% \item Interest unpaid in a period on a class of notes is not made up in a following period.
% \end{itemize}

% This approach captures the principal feature of the trade while omitting fees, IC, OC tests and catch-up interest payments.
% The MonteCarlo implementation is less restrictive and allows for fees and coverage tests.

% First we calculate the notionals, $N_k(t)$. To do this we need the distributions of the redemptions on a single asset which is given by
% $$
% Red^j(t)=\left\{ 
% \begin{array}{ll} 
% R^jN^j{\bf1} _{\{\tau^j< t\} }, & \hspace{0.1 in} t<t^j_n, \\
% R^jN^j{\bf1} _{\{\tau^j < t^j_n\} }+N^j{\bf1} _{\{\tau^j\geq t^j_n\}}, & \hspace{0.1 in} t\geq t^j_n. 
% \end{array} 
% \right.
% $$
% This can be expressed using indicator functions as
% $$
% Red^j(t) =R^jN^j{\bf1} _{\{\tau^j< t\} }{\bf1} _{\{t<t^j_n\} }
% +N^j\left[ R^j{\bf1} _{\{\tau^j < t^j_n\} }+{\bf1} _{\{\tau^j\geq t^j_n\}}
% \right]{\bf1} _{\{t\geq t^j_n\} },
% $$
% where $\tau^j$ is the time to default of the jth asset and $R^j$ the recovery rate on the $j$th asset and $t_n^j$ the maturity date of the $j$th asset. The distribution of the portfolio redemption to time $t$ is thus calculated as
% $$
% Red(t)=\sum_j Red^j(t).
% $$

\subsection*{MonteCarlo algorithm}

A systematic numerical way of evaluation the time to default of the redemption flows is a MonteCarlo simulation. 
Correlated default times are generated by drawing correlated Gaussian random variates for each sample path and each underlying bond / loan. 
Those variates, e.g. $y$, are transformed into default times exploiting the following relationship: 

$$
\tau = \frac{ ln \left( 1- N \left( y \right) \right) } { h \left( t \right) } 
$$

The redemption / principal flows together with the interest flows are assigned into a time grid created from the tranche’s payment dates. 
Those collections are allocated subsequently on each MC path in each time step to through the following waterfall structure to the tranches.
For simplicity a two tranche case is shown, more tranche are possible. 

\begin{itemize}
	\item Senior fees (paid from interest collections)
	\item Interest due on senior notes (paid from interest collections)
	\item Redemption on senior notes, upon coverage test breach (paid from interest collections)
	\item Interest due on junior notes (paid from interest collections)
	\item Redemptions on senior notes until paid in full (paid from principal collections)
	\item Redemptions on junior notes until paid in full (paid from principal collections)
	\item Junior fees (paid from interest collections)
	\item Residual payments split between junior note and incentive fee (paid from both collections)
\end{itemize}

These tranche cashflows can be discounted to calculate the NPV for each tranche in each MC path. 
Finally, the reported NPV is the average of all path NPVs. 
