%========================================================
\section{Launchers and Visualisation}\label{sec:visualisation}
%========================================================

\subsection{Jupyter}\label{sec:jupyter}

ORE comes with an experimental Jupyter notebook for launching ORE batches and in particular for drilling into NPV cube
data.  The notebook is located in directory {\tt FrontEnd/Python/Visualization/npvcube}. To launch the notebook, change
to this directory and follow instructions in the {\tt Readme.txt}. In a nutshell, type\footnote{With Mac OS X, you may
  need to set the environment variable {\tt LANG} to {\tt en\_US.UTF-8} before running jupyter, as mentioned in the
  installation section \ref{sec:python}.}

\medskip
\centerline{\tt jupyter notebook}
\medskip

to start the ipython console and open a browser window. From the list of files displayed in the browser then click

\medskip
\centerline{\tt ore\_jupyter\_dashboard.ipynb} 
\medskip

to open the ORE notebook. The notebook offers
\begin{itemize}
\item launching an ORE job
\item selecting an NPV cube file and netting sets or trades therein
\item plotting a 3d exposure probability density surface
\item viewing exposure probability density function at a selected future time
\item viewing expected exposure evolution through time  
\end{itemize}

The cube file loaded here by default when processing all cells of the notebook (without changing it or launching a ORE
batch) is taken from {\tt Example\_7} (FX Forwards and FX Options).

%\todo[inline]{Add Jupyter section}

\subsection{Calc}\label{sec:calc}

ORE comes with a simple LibreOffice Calc \cite{LO} sheet as an ORE launcher and basic result viewer. This is
demonstrated on the example in section \ref{example:exposure_swapflat}. It is currently based on the stable LibreOffice version 5.0.6
and tested on OS X. \\

To launch Calc, open a terminal, change to directory {\tt Examples/Example\_1}, and run

\medskip
{\centerline{\tt ./launchCalc.sh} }
\medskip

%This will show the blank sheet in figure \ref{fig_14}.
%\begin{figure}[h]
%\begin{center}
%\includegraphics[scale=0.4]{demo_calc_1}
%\end{center}
%\caption{Calc sheet after launching.}
%\label{fig_14}
%\end{figure}
The user can choose a configuration (one of the {\tt ore*.xml} files in Example\_1's subfolder Input) by hitting the
'Select' button. Initially Input/ore.xml is pre-selected. The ORE process is then kicked off by hitting 'Run'. Once
completed, standard ORE reports (NPV, Cashflow, XVA) are loaded into several sheets. Moreover, exposure evolutions can
then be viewed by hitting 'View' which shows the result in figure \ref{fig_16}.  \\
\begin{figure}[h]
\begin{center}
\includegraphics[scale=0.4]{demo_calc_2}
\end{center}
\caption{Calc sheet after hitting 'Run'.}
\label{fig_16}
\end{figure}

This demo uses simple Libre Office Basic macros which call Python scripts to execute ORE. The Libre Office Python uno
module (which comes with Libre Office) is used to communicate between Python and Calc to upload results into the sheets.

%\todo[inline]{Remove hard-coded file names from Python scripts}
%\todo[inline]{Calc example on Windows and Linux} 
%\todo[inline]{Harmonise layout with Excel launcher} 

\subsection{Excel}\label{sec:excel}

ORE also comes with a basic Excel sheet to demonstrate launching ORE and presenting results in Excel. This demo is more
self-contained than the Calc demo in the previous section, as it uses VBA only rather than calls to external Python
scripts. The Excel demo is available in Example\_1. Launch {\tt Example\_1.xlsm}. Then modify the paths on the first
sheet, and kick off the ORE process.

